%! Author = louis
%! Date = 28-10-22

% Preamble
\documentclass[12pt]{article}

% Packages
\usepackage{amsmath}
\usepackage{lstdoc}
\usepackage{hyperref}
\usepackage{pdfpages}
\usepackage{graphicx}
\hypersetup{
    colorlinks=true,
    linkcolor=black,
    filecolor=magenta,
    urlcolor=cyan,
    pdftitle={Clever-Party-Thrower, une application web hébergée via Kubernetes et Ansible},
    pdfpagemode=FullScreen,
}


% Document
\begin{document}
    \begin{titlepage}
        \begin{center}
            \vspace*{1cm}

            \Huge
            \textbf{Clever-Party-Thrower, une application web hébergée via Kubernetes et Ansible}

            \vspace{0.5cm}

            \vspace{1.5cm}

            \textbf{Louis De Wilde}

            \vfill

            \includegraphics[width=0.65\textwidth]{images/EPHEC_logo.svg}
            \vfill
            \vspace{0.8cm}
            \Large
            Technologie de l'informatique 3TL2 \\
            Ephec : Av. du Ciseau 15, 1348 Ottignies-Louvain-la-Neuve\\
            Novembre 2022

        \end{center}
    \end{titlepage}
    \thispagestyle{empty}
    \newpage
    \setcounter{page}{0}
    \begin{KeepFromToc}
        \tableofcontents
    \end{KeepFromToc}
    \newpage
   % \bibliography{main}
    %\bibliographystyle{plain}

    \section{Cahier des charges}\label{sec:cahier-des-charges}
    %! Author = louis
%! Date = 28-10-22

\subsection{Choix du sujet}\label{subsec:choix-du-sujet}
En tant qu’étudiant (ou personnes aimant nous amuser de manière générale), nous sommes souvent amenés à organiser une soirée ou une fête entre amis.
Il faut alors trouver une date qui convient au au plus grand nombre et gérer les dépenses de la fête, ce qui peut s'avérer compliqué quand le nombre de participants augmente.
C’est pourquoi j’ai décidé de créer une application web facilitant l’organisation de petites fêtes estudiantines.

\subsection{Cahier des charges}\label{subsec:cahier-des-charges}

\subsubsection{Besoin du client}
\begin{itemize}
    \item Permet de répartir équitablement les frais engendrés par la fête
    \item Permet de créer une playlist commune (sur laquelle chacun a la possibilité d'ajouter des titres)
    \item Crée une liste des courses pour la soirée (éventuellement une répartition si toutes les courses ne se font pas au meme endroit ?)
    \item Connaître qui a amène quoi à la fête
    \item Crée différents évènements
    \item Est accessible grâce à un simple lien
    \item Est pourvu d'un système de connexion et de création de compte simple
    \item Permet de trouver une date d’évènement qui convient au plus grand nombre de participants
    \item Permet au participant d'organiser facilement des covoiturages
    \item Permet au conducteur de connaitre son trajet de covoiturage
    \item Calcule et reparti automatiquement les frais de transport entre les différents covoitureurs
\end{itemize}

\subsubsection{Contraintes}
\begin{itemize}
    \item Accessible sur toutes les plateformes (Android, IOS, Mac, Windows, Linux)
    \item Le système de création de compte doit être simple
    \item Les données des utilisateurs seront sécurisées
    \item L’application doit être auto-hébergable
\end{itemize}

\subsubsection{Méthodologie}
N’ayant pas de client autre que moi, je ne compte pas appliquer de méthodologie vraiment spécifique (comme scrum par exemple).\\
En revanche, je vais m’inspirer de cette dernière en divisant les demandes de l’utilisateur en petites tâches.\\
Afin de suivre l’avancement du projet et de m’organiser au mieux, j’utiliserai FocalBoard, qui me servira à créer des cartes représentant les différentes tâches et me permettra de les gérer visuellement.\\
Le code, quant à lui, sera géré sur \href{https://github.com/craftbrak/Clever-Party-Thrower}{Github}\footnote{\url{https://github.com/craftbrak/Clever-Party-Thrower}}.
Je voudrais si possible, pour certaines parties du projet, utiliser de l'intégration voire du déploiement continu, en plus de tests automatisés.
    \newpage

    \section{Analyse}\label{sec:analyse}
    %! Author = louis
%! Date = 28-10-22

Le projet ayant comme contrainte pricipale d'etre disponible sur toute les platformes majeur une base web me parait évident.
pourtant le web viens aussi avec la limitation pricipale qu'est le besoin de connection permanente au serveur,
pour remedier a ce probleme un PWA (Application Web Progressive) nous permetrais de beneficiez des avantages du web,
tout en permetant aux utilisateurs de concerver une version locale de l'application.
De plus une PWA permet d'etre instalée comme une application native Android ou IOS pour les platformes mobiles ce qui facilite l'accès.\\\\

Une PWA comme toute application web necesite plusieurs composant:
\begin{itemize}
    \item Un back-end
    \item une Base de Donnée
    \item une Interface definie pour le back-end (API)
    \item un front-end
\end{itemize}
\subsection{Back-end}\label{subsec:back-end}
Le serveur back-end doit etre capable de mettre a disposition les donnée nescesaire au fonctionnement de l'application.
Cependant le back-end sera aussi responcable des calcul d'equilibrage des depence , des matching de covoiturage et des trajets.
Connaissant deja Express.js et Spring (Kotlin) je me suis naturellement tourner vers ce derniers néanmoins je desirais utiliser du TypeScript pour ce projet.
Express.js est compatible avec TypeScript mais Spring ne l'est pas, Express.js est leger, minimaliste et propose une bonne documentation,
Spring est generaliste propose des outils de generation de code et utilise une systeme d'injection de dependance.
Ce qu'il me falais etait une hybride entre Spring et Express.js.
Apres quelque recherche j'ai trouver un framework appler Nest.js, ce dernier supporte nativement TypeScript, propose une documentation ecxelante,
propose un CLI permetant de generer du code boiler plate, reste simple et leger, Utilise de l'injection de dependance et est tres utiliser et apprecier par d'autre developeurs.
J'ai donc decider d'utiliser Nest.Js pour ce projet.

\subsubsection{ORM}
Nest.Js (que je vais appeller Nest a partir de maintenant) n'est pas un framework batteries included comme le son Spring ou Larravel,
Il m'a donc fallu choisir aussi un ORM pour interfacer avec ma base de donnée.
Dans la Documentation de Nest, il y a des exemple de configuration avec les ORM les plus Utiliser avec Nest et TypeScrip.
Ces derniers sont Sequelize, TypeOrm et Prisma ,
ayant eu une experience avec Sequelize plus tot mauvaise lors de mon projet de dev3, j'ai decider de regarder du coté de prisma.
Prisma est une ORM un peu particulier dans le sens ou il propose d'utiliser une schema prisma pour definir ces entité plutot que le code TypeScript.
Cella me parraissait une tres bonne chose vu que prisma generais alors lui meme les differents DTO nescesaire.
Il c'est cependant averer que a l'utilisation prisma n'est pas l'ideale surtout lorsque l'on desire le coupler a une API graphQl
A cause de son schema particulier je devais definir le forma de mes entitée a 2 endroit et tout de meme crée manuelement mes DTO
J'ai donc decider de jeter une oeuil a TypeOrm, celui ci propose de definir les entité via des anotation TypeScript.
Ce qui me permet d'utiliser une seule classe pour definir mon entitée d'orm et de graphQl de plus TypeOrm est mieux
documenter grace au support d'une comunauter plus large.


\subsection{La base de donnée}\label{subsec:la-base-de-donnee}
Ma base de donnée devais me permetre 2 chose, stocker des donnée geographique et faire des operation sur ces donnée.
Le seul choix qui s'offrais a moi compatible avec TypeOrm etait postgress avec une extention nomer POSTGIS.
Postgres est une base de donnée tres utiliser en production et est tres performante et stable.

\subsection{L'API back end}\label{subsec:l'api-back-end}
Il existe plusieurs type d'API tres utiliser pour le Back-end mais les plus populaire sont le REST et le SOAP.
SOAP est de moin en moin utiliser du a la complexité intrinsec du xml et a ca lourdeur.
Il nous reste donc REST qui est une bonne solution mais nescecite pour presque chaque requete de recuperer des donnée qui ne seront pas toujour consomée par le client
Avec Rest l'on recupere une entitée complete tout les champs sont renvoyer.
C'est pour cela que graphQl a ete crée , non seulement graphQL utilise un typage fort via un schema defini et exposer par le serveur,
mais en plus graphql permet de ne recuperer que les champs nescesaire au client grace au GQL.




\subsection{Front-end}\label{subsec:front-end}
Pour ce qui est de l'interface utilisateur ( le front end ) j'ai decidé d'utiliser angular car ce framework utilise TypeScript nativement et supporte aussi les PWA
TypeScript est important car il permet de eliminer les bugs lier au typage d'objet avant meme l'execution du programe.
Angular etant un framework orienter m'oblige aussi à suivre une architecture propre qui permet de rendre independant les differents composants du projet.



    %\backend
    % La db (shema+ ORM + GDPR)
    % graphql
    % NestJs

    % Le front end
    % agular
    % apollo
    % les service
    % l'authentification par token

    \section{hebergement}
    % proxmox
    % ansible
    % kubernetes
    % traefik, certmanager & https
    % longhorn & rancher
    % reseau (ce qui aurais du etre fait et ce qui est fait)
    \newpage
\end{document}
