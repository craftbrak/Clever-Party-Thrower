\subsection{Introduction}\label{subsec:introduction}
Cette section introduit l'architecture, la documentation du code, les bonnes pratiques de programmation, les tests unitaires et la sécurité de l'application web Clever Party Thrower.

\subsection{Architecture}\label{subsec:architecture}
Clever Party Thrower est une application web full-stack.
Le front-end est construit avec Angular tandis que le back-end est un monolithe élaboré avec NestJS et TypeORM. À chaque accès au site par un utilisateur,
la Single Page Application (SPA) Angular est chargée, initiant les services nécessaires pour interagir avec le back-end.
Ces services récupèrent les données requises via l'API GraphQL, en utilisant Apollo.

Durant toute la session de l'utilisateur, ces services persistent et continuent de récupérer les données du back-end au fur et à mesure de la navigation.
Chaque service est responsable d'un type de données spécifique.
Par exemple, AuthService s'occupe des données liées à l'authentification et à l'utilisateur, tandis qu'EventService gère tout ce qui est directement lié à un événement.

De la même manière, le back-end est divisé en plusieurs services, chacun gérant une ressource unique.
Ces services sont rassemblés en modules, associés aux contrôleurs REST et aux résolveurs GraphQL. Cette modularité facilite le test du code en assurant un découplage
entre les différents services.

\subsection{Documentation du code}\label{subsec:documentation-du-code}
Le code de Clever Party Thrower est organisé et documenté pour faciliter sa compréhension et sa maintenance.
Grâce à TypeScript, la majorité de la documentation est implicite.
Les noms des fonctions et leurs arguments offrent une compréhension rapide de leur utilité.
Pour les fonctions plus complexes, des commentaires expliquent la logique algorithmique.
Ainsi, même un développeur novice dans le projet peut facilement naviguer dans le code.

\subsection{Bonnes pratiques de programmation}\label{subsec:bonnes-pratiques-de-programmation}
Clever Party Thrower suit les conventions de codage d'Angular et de NestJS, ce qui comprend l'utilisation appropriée des indentations,
des commentaires, de la casse des lettres (camelCase, PascalCase), et des conventions de nommage des variables, des fonctions et des classes.
Cette pratique améliore la lisibilité du code et facilite le travail en équipe.

De plus, le code est divisé en modules et services distincts, chacun étant responsable d'une fonctionnalité spécifique.
Cela favorise la modularité, permet une meilleure gestion du code et facilite la maintenance et l'extension de l'application.

Les principes DRY (Don't Repeat Yourself) et SOLID sont également respectés pour minimiser la duplication de code et assurer la prédictabilité du comportement du logiciel.
Quelques concepts de programmation fonctionnelle sont utilisés dans le front-end, tels que l'immutabilité et l'utilisation maximale de fonctions pures.

La gestion des erreurs en JavaScript et TypeScript est plus basique que dans d'autres langages de programmation.
Contrairement à Java, où les erreurs doivent être déclarées dans la définition de la fonction, TypeScript ne possède pas cette fonctionnalité.
Il est donc difficile de savoir quelle fonction peut lancer une erreur.
La meilleure pratique est d'utiliser des blocs try-catch au niveau le plus élevé de l'application pour capturer les erreurs une fois qu'elles sont propagées.

\subsection{Tests Unitaires}\label{subsec:tests-unitaires}
Les tests unitaires sont une partie essentielle de Clever Party Thrower.
Chaque module et service dispose de ses propres tests unitaires, garantissant ainsi que chaque partie de l'application fonctionne comme prévu.
Les tests sont écrits avec Jest, un framework de test populaire pour JavaScript et TypeScript.

Le découplage des services et des modules rend les tests plus simples et plus efficaces, car chaque test se concentre sur une seule unité de code.

\subsection{Sécurité}\label{subsec:securite}
La sécurité est une préoccupation majeure pour Clever Party Thrower.
Parmi les mesures de sécurité mises en place, on compte l'utilisation de HTTPS pour toutes les communications via un reverse proxy dans l'infrastructure d'hébergement,
l'authentification et l'autorisation basées sur des tokens JWT, la validation des entrées côté serveur pour prévenir les attaques par injection,
et le chiffrement des données sensibles comme les mots de passe avant leur enregistrement en base de données.
Les vulnérabilités potentielles sont régulièrement évaluées et des mises à jour de sécurité sont appliquées dès que nécessaire via npm audit et npm audit fix.