\subsection{État Actuel du Produit}\label{subsec:etat-actuel-du-produit}

Le produit actuel est une version minimale viable (MVP) de l'application web "Clever Party Thrower".
Il s'agit d'une application en cours de développement, mais qui dispose déjà de plusieurs fonctionnalités clés, dont l'envoi d'invitations,
la gestion des participants, le mécanisme de dettes, le vote pour les dates et la liste de courses.\\

Le système de covoiturage est presque achevé, mais manque encore de l'interface et du système de calcul de distance.
Des fonctionnalités telles que la double authentification et le mécanisme d'email ne sont pas encore opérationnelles.

\subsection{Installation du Produit}\label{subsec:installation-du-produit}

Une procédure d'installation détaillée est fournie dans le fichier README sur le dépôt GitHub.
Cependant, pour une compréhension rapide,
voici une vue d'ensemble des étapes nécessaires pour installer le produit :

\begin{enumerate}
    \item Prérequis : Docker doit être installé sur la machine.
    \item Étapes d'installation :
    \begin{enumerate}
        \item Téléchargez le fichier docker-compose disponible sur le dépôt GitHub et placez-le sur la machine hôte de l'application.
        \item Modifiez les variables d'environnement comme indiqué dans le fichier ou créez un nouveau fichier d'environnement avec les valeurs appropriées.
        \item Lancez la commande : \texttt{docker-compose up -d}.
    \end{enumerate}
\end{enumerate}

Pour plus de détails ou pour résoudre les problèmes d'installation, veuillez consulter le fichier README sur le dépôt GitHub.

\subsection{Démonstration du Produit}\label{subsec:demonstration-du-produit}

Une démonstration du produit a été préparée pour montrer son fonctionnement.
Cette démonstration met en évidence les fonctionnalités existantes de l'application, y compris la création d'un nouvel événement et la gestion des invitations.\\

Malgré le fait que l'application ne soit pas encore complète, cette démonstration donne un aperçu de ce qu'elle pourra accomplir une fois finalisée.
En outre, les retours positifs reçus jusqu'à présent indiquent que l'application répond déjà à
un besoin réel et qu'elle a le potentiel de devenir un outil précieux pour l'organisation de fêtes.

\subsection{Challenges et Améliorations Futures}\label{subsec:challenges-et-ameliorations-futures}

Il y a eu plusieurs défis lors du développement de ce projet.
L'un des plus notables a été l'incapacité à héberger l'application sur un cluster Kubernetes géré via Ansible,
malgré de nombreux efforts pour résoudre ce problème.
C'est une amélioration importante à réaliser à l'avenir, car elle permettra une haute disponibilité de l'application.\\

D'autres améliorations comprennent la finalisation du système de covoiturage, l'introduction de la double authentification et du mécanisme d'email,
et l'amélioration de l'interface utilisateur.

\subsection{Points de Fierté}\label{subsec:points-de-fierte}

Je suis particulièrement fier de la qualité du code sur le back-end et sur le front-end.
J'ai investi beaucoup de temps et d'efforts pour m'assurer que le code soit propre, bien documenté et conforme aux bonnes pratiques de programmation.
Je suis convaincu que ces efforts paieront sur le long terme, en facilitant la maintenance et l'évolutivité de l'application.