\subsection{Analyse critique du projet}\label{subsec:analyse-critique}

Le développement de ce projet a mis en lumière divers points forts et quelques zones à améliorer.\\

\textbf{Points forts:}
\begin{itemize}
    \item \textbf{Qualité du code :} Un investissement conséquent en temps a été consacré à produire un code de qualité, bien structuré, documenté et conforme aux meilleures pratiques de programmation.
    \item \textbf{CI/CD :} L'intégration et le déploiement continus ont démontré leur valeur en contribuant au maintien de la qualité du code et en accélérant le processus de développement.
    \item \textbf{TypeOrm :} L'utilisation de TypeOrm a facilité les itérations rapides sur la structure des données.
    \item \textbf{NestJs :} Le choix d'utiliser NestJs a permis d'ajouter rapidement des fonctionnalités au back-end.
    \item \textbf{Angular :} L'utilisation d'Angular a facilité la construction rapide d'une interface tout en simplifiant les tests.
\end{itemize}\\

\textbf{Points faibles:}
\begin{itemize}
    \item \textbf{Infrastructure :} L'implémentation de l'infrastructure, en particulier l'hébergement sur un cluster Kubernetes géré via Ansible, a représenté un défi majeur.
    \item \textbf{Fonctionnalités non terminées :} Certaines fonctionnalités, telles que le système de covoiturage et le mécanisme d'email, ne sont pas encore opérationnelles.
    \item \textbf{Interface utilisateur :} Certains composants de l'interface utilisateur pourraient être améliorés grâce à des animations.
\end{itemize}

\subsection{Améliorations envisageables}\label{subsec:ameliorations}

Le projet pourrait être amélioré à l'avenir dans plusieurs domaines.\\

\begin{itemize}
    \item \textbf{Infrastructure :} Résoudre le problème d'hébergement sur le cluster Kubernetes géré via Ansible serait bénéfique.
    \item \textbf{Fonctionnalités :} Il serait avantageux de finaliser le système de covoiturage et de mettre en place la double authentification ainsi que le mécanisme de validation d'email.
    \item \textbf{Interface utilisateur :} L'interface utilisateur pourrait être améliorée pour une meilleure convivialité, peut-être en adoptant un framework d'interface utilisateur plus moderne.
\end{itemize}

\subsection{Plan pour le futur}\label{subsec:plan-futur}

Dans le futur, je prévois de continuer à développer et améliorer ce projet.
Mon intention est de résoudre les problèmes d'infrastructure existants, de terminer les fonctionnalités inachevées et d'améliorer l'interface utilisateur en ajoutant par exemple plus d'animation et un système de traduction.
J'envisage également d'ajouter des fonctionnalités telles qu'une playlist Spotify partagée pour l'événement, un chat instantané, et un mécanisme de partage d'images.

\subsection{Ce que j'aurais fait autrement}\label{subsec:ce-que-j-aurais-fait-autrement}

Avec le recul, j'aurais abordé certains aspects du projet différemment.
J'aurais, par exemple, commencé à travailler sur l'infrastructure plus tôt dans le processus de développement pour éviter certains des problèmes rencontrés.\\

De plus, je remets en question le choix de TypeScript pour le développement de l'application.
Ayant eu l'occasion d'apprendre Go et de découvrir les principes de Rust, je suis convaincu que l'utilisation de l'un de ces langages aurait pu faciliter le développement,
notamment en ce qui concerne la gestion des erreurs.\\

De plus, bien que NestJs présente de nombreux points forts, il souffre des limitations de l'environnement JavaScript et du manque de typage fort sur plusieurs aspects.
Si je n'avais pas utilisé Go ou Rust, mon deuxième choix aurait été Kotlin avec Spring.
Comme mentionné dans mon analyse, cette combinaison offre de nombreux avantages.\\


En conclusion, je réalise que le développement d'une application nécessite une planification détaillée et une exécution rigoureuse.
Sur un plan plus personnel, je constate qu'il est bénéfique pour moi de me fixer de petits objectifs à court terme pour optimiser ma productivité.
À l'avenir, je m'efforcerai d'appliquer les leçons apprises de ce projet pour améliorer mes futurs efforts de développement.