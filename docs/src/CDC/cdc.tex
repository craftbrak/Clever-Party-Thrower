%! Author = louis
%! Date = 28-10-22

\subsection{Choix du sujet}\label{subsec:choix-du-sujet}
En tant qu’étudiant (ou personnes aimant nous amuser de manière générale), nous sommes souvent amenés à organiser une soirée ou une fête entre amis.
Il faut alors trouver une date qui convient au au plus grand nombre et gérer les dépenses de la fête, ce qui peut s'avérer compliqué quand le nombre de participants augmente.
C’est pourquoi j’ai décidé de créer une application web facilitant l’organisation de petites fêtes estudiantines.

\subsection{Cahier des charges}\label{subsec:cahier-des-charges}

\subsubsection{Besoin du client}
\begin{itemize}
    \item Permet de répartir équitablement les frais engendrés par la fête
    \item Permet de créer une playlist commune (sur laquelle chacun a la possibilité d'ajouter des titres)
    \item Crée une liste des courses pour la soirée (éventuellement une répartition si toutes les courses ne se font pas au meme endroit ?)
    \item Connaître qui a amène quoi à la fête
    \item Crée différents évènements
    \item Est accessible grâce à un simple lien
    \item Est pourvu d'un système de connexion et de création de compte simple
    \item Permet de trouver une date d’évènement qui convient au plus grand nombre de participants
    \item Permet au participant d'organiser facilement des covoiturages
    \item Permet au conducteur de connaitre son trajet de covoiturage
    \item Calcule et reparti automatiquement les frais de transport entre les différents covoitureurs
\end{itemize}

\subsubsection{Contraintes}
\begin{itemize}
    \item Accessible sur toutes les plateformes (Android, IOS, Mac, Windows, Linux)
    \item Le système de création de compte doit être simple
    \item Les données des utilisateurs seront sécurisées
    \item L’application doit être auto-hébergable
\end{itemize}

\subsubsection{Méthodologie}
N’ayant pas de client autre que moi, je ne compte pas appliquer de méthodologie vraiment spécifique (comme scrum par exemple).\\
En revanche, je vais m’inspirer de cette dernière en divisant les demandes de l’utilisateur en petites tâches.\\
Afin de suivre l’avancement du projet et de m’organiser au mieux, j’utiliserai FocalBoard, qui me servira à créer des cartes représentant les différentes tâches et me permettra de les gérer visuellement.\\
Le code, quant à lui, sera géré sur \href{https://github.com/craftbrak/Clever-Party-Thrower}{Github}\footnote{\url{https://github.com/craftbrak/Clever-Party-Thrower}}.
Je voudrais si possible, pour certaines parties du projet, utiliser de l'intégration voire du déploiement continu, en plus de tests automatisés.