\subsection{Contexte}
Dans le contexte des étudiants et des personnes qui aiment s'amuser, l'organisation de petites fêtes estudiantines est souvent une tâche complexe.
Il est nécessaire de prendre en compte la date, les dépenses, la logistique et les préférences des participants.
Pour faciliter ce processus, une application web est proposée pour assister les organisateurs et les participants dans la gestion des différents aspects de l'événement.

\subsection{Besoin du client}\label{subsec:besoin-du-client}
\begin{itemize}
    \item Permet de répartir équitablement les frais engendrés par la fête
    \item Crée une liste des courses pour la soirée (éventuellement une répartition si toutes les courses ne se font pas au meme endroit ?)
    \item Connaître qui a amène quoi à la fête
    \item Crée différents évènements
    \item Est accessible grâce à un simple lien
    \item Est pourvu d'un système de connexion et de création de compte simple
    \item Permet de trouver une date d’évènement qui convient au plus grand nombre de participants
\end{itemize}

\subsection{Fonctionnalités optionnelles}\label{subsec:fonctionnalites-optionnelles}
\begin{itemize}
    \item Permet de créer une playlist commune (sur laquelle chacun a la possibilité d'ajouter des titres)
    \item Permet au conducteur de connaitre son trajet de covoiturage
    \item Permet au participant d'organiser facilement des covoiturages
    \item Calcule et reparti automatiquement les frais de transport entre les différents covoitureurs
\end{itemize}

\subsection{Contraintes}\label{subsec:contraintes}
\begin{itemize}
    \item Accessible sur toutes les plateformes (Android, IOS, Mac, Windows, Linux)
    \item Le système de création de compte doit être simple
    \item Les données des utilisateurs seront sécurisées
    \item L’application doit être auto-hébergable
    \item Respect des réglementations en matière de protection des données (GDPR)
\end{itemize}

\subsection{Méthodologie}\label{subsec:methodologie}
Pour la réalisation du projet, une méthodologie inspirée de Scrum sera utilisée, avec la division des demandes de l'utilisateur en petites tâches.
L'outil FocalBoard sera utilisé pour organiser et suivre l'avancement des tâches.
Le code sera géré sur GitHub, avec l'intention d'utiliser l'intégration et le déploiement continus, ainsi que des tests automatisés pour certaines parties du projet.
L'étudiant travaillera en étroite collaboration avec le rapporteur pour valider les différentes étapes du projet et s'assurer que les attentes sont respectées.

\subsection{Interactions avec le client (et/ou le rapporteur)}\label{subsec:interactions-avec-le-client-(et/ou-le-rapporteur)}
L'étudiant fournira des mises à jour régulières au rapporteur sur l'avancement du projet, et sollicitera des conseils et des orientations si nécessaire.
Des réunions périodiques pourront être organisées pour discuter des problèmes, des progrès et des améliorations possibles.