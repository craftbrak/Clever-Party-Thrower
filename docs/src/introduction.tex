

Dans le monde d'aujourd'hui, les étudiants et les personnes en général sont souvent amenés à organiser des événements, tels que des soirées ou des fêtes entre amis.
Cependant, gérer les aspects logistiques de ces événements peut s'avérer compliqué, notamment lorsqu'il s'agit de fixer une date, de répartir les coûts et de coordonner les participants.
Afin de faciliter ce processus et d'améliorer l'expérience de l'organisation de fêtes, j'ai décidé de développer une application web nommée "Clever Party Thrower".
Cette application permettra aux utilisateurs de gérer efficacement tous les aspects importants de l'organisation d'un événement,
y compris les dépenses, les courses voir meme la musique, les covoiturages et bien plus encore.\\\\

Dans ce rapport, je présenterai les différentes étapes du développement de l'application Clever Party Thrower,
en commençant par le choix du sujet et en détaillant le cahier des charges, l'analyse de la problématique, la conception et la réalisation du projet.
J'aborderai également les aspects liés à la sécurité, la gestion des versions, les tests et la documentation du code.\\
Un accent particulier sera mis sur l'adoption de pratiques de développement modernes, telles que l'intégration continue (CI) et le déploiement continu (CD), afin d'assurer la qualité et la fiabilité du code tout au long du cycle de développement.
De plus, je discuterai des choix d'hébergement pour l'application, en explorant les options d'auto-hébergement voir meme compatible avec le cloud.\\

Enfin, je terminerai en présentant une conclusion sur le travail accompli, les défis rencontrés et les perspectives d'amélioration pour le futur.