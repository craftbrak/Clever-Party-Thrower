Les ressources suivantes ont été déterminantes dans le développement de ce projet :

\subsection{Outils de développement}\label{subsec:outils-de-developpement}
\begin{itemize}
    \item \textbf{Docker~\cite{dockerDoc:online} :} A facilité la conteneurisation de l'application, optimisant ainsi la gestion et le déploiement.
    \item \textbf{Ansible~\cite{ansibleDoc:online} :} A permis l'automatisation de la gestion des configurations, rendant le déploiement de l'infrastructure plus aisé.
    \item \textbf{GitHub~\cite{craftbra55:online} :} A servi pour le contrôle de version, la gestion de projets et comme dépôt de code.
    \item \textbf{Github~\cite{Usingwor49:online} workflows :} A garanti un cycle de développement plus fluide grâce à l'intégration et le déploiement continus.
\end{itemize}

\subsection{Langages de programmation}\label{subsec:langages-de-programmation}
\begin{itemize}
    \item \textbf{TypeScript~\cite{TypeScri50:online} :} Langage principal pour le développement de l'application.
    \item \textbf{HTML/CSS/SCSS~\cite{SassSynt24:online} :} Impliqués dans la conception de l'interface utilisateur.
    \item \textbf{YAML/Jinja~\cite{JinjaThe11:online} :} Employés pour la configuration de Kubernetes et dans le playbook Ansible.
    \item \textbf{Markdown~\cite{Markdown57:online} :} Utilisé pour la création des différents fichiers Readme.
    \item \textbf{LaTex~\cite{LaTeX—Wi69:online} :} Utilisé pour la rédaction de ce rapport.
\end{itemize}

\subsection{Frameworks et bibliothèques JavaScript}\label{subsec:frameworks}
\begin{itemize}
    \item \textbf{NestJs~\cite{NestJSAp67:online} :} Facilité le développement du back-end de l'application.
    \item \textbf{Angular~\cite{Angular86:online} :} Employé pour construire l'interface utilisateur et mettre en oeuvre les tests.
    \item \textbf{TypeOrm~\cite{TypeORMA41:online} :} Simplifié la gestion de la base de données.
    \item \textbf{PassportJs~\cite{passport9:online} :} Utilisé pour la gestion de l'authentification.
    \item \textbf{Material UI~\cite{AngularM50:online} :} Utilisé pour créer les composants de l'interface sur le front-end.
    \item \textbf{Apollo Angular~\cite{Home–Ang72:online} :} Permis la communication avec l'API GraphQL du back-end.
    \item \textbf{Apollo Server~\cite{Introduc33:online} :} A facilité la création de l'API GraphQL et la gestion des requêtes sur le back-end.
\end{itemize}

\subsection{APIs tierces}\label{subsec:apis-tierces}
\begin{itemize}
    \item \textbf{Open Street Map~\cite{API—Open50:online} :} Cette API sera utilisée pour calculer les distances pour les covoiturages.
    \item \textbf{Spotify~\cite{WebAPISp43:online} :} Potentiellement utilisée dans le futur.
    \item \textbf{Dicebear~\cite{JSLibrar55:online} :} Utilisée pour générer des avatars pour les utilisateurs.
    \item \textbf{RestCountries~\cite{RESTCoun99:online} :} Utilisée pour remplir la base de donnees avec tout les pays disponible.
\end{itemize}

\subsection{Autres ressources}\label{subsec:autres-ressources}
\begin{itemize}
    \item \textbf{Stack Overflow~\cite{StackOve24:online} et Stack Exchange~\cite{HotQuest86:online} :} Sources précieuses de connaissances pour résoudre des problèmes spécifiques.
    \item \textbf{Documentations officielles :} Les documentations officielles d'Angular, NestJs, TypeOrm, Material UI, Apollo Angular, et K3S ont été des guides essentiels.
    \item \textbf{Chaîne YouTube de TechnoTim~\cite{1TechnoT44:online} :} Principale source d'inspiration pour le cluster Kubernetes.
    \item \textbf{Dépôt Git de TechnoTim~\cite{technoti52:online} :} Ce dépôt a servi de base pour la création du cluster Kubernetes.
\end{itemize}

Ces ressources ont joué un rôle crucial dans la réalisation de ce projet, contribuant à chaque étape du processus de développement.s de développement.

\subsection{Ressources du projet}\label{subsec:ressources-du-projet}
\begin{itemize}
    \item \textbf{Dépôt github de l'application~\cite{craftbra13:online} :} Le code source de l'application est hébergée sur ce Dépôt git.
    \item \textbf{Dépôt github de l'infrastructure~\cite{craftbra86:online} :} Le code source de l'infrastructure est hébergée sur ce Dépôt git.
    \item \textbf{Mon hébergement de l'application~\cite{CleverPa16:online} :} l'application est hébergée et disponible a cet url.
\end{itemize}