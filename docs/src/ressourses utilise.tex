Les ressources suivantes ont été déterminantes dans le développement de ce projet :

\subsection{Outils de développement}\label{subsec:outils-de-developpement}
\begin{itemize}
    \item \textbf{Docker :} A facilité la conteneurisation de l'application, optimisant ainsi la gestion et le déploiement~\cite{DockerDoc:online}.
    \item \textbf{Ansible :} A permis l'automatisation de la gestion des configurations, rendant le déploiement de l'infrastructure plus aisé~\cite{AnsibleDoc:online}.
    \item \textbf{GitHub :} A servi pour le contrôle de version, la gestion de projets et comme dépôt de code.
    \item \textbf{Github workflows :} A garanti un cycle de développement plus fluide grâce à l'intégration et le déploiement continus.
\end{itemize}

\subsection{Langages de programmation}\label{subsec:langages-de-programmation}
\begin{itemize}
    \item \textbf{TypeScript :} Langage principal pour le développement de l'application.
    \item \textbf{HTML/CSS/SCSS :} Impliqués dans la conception de l'interface utilisateur.
    \item \textbf{YAML/Jinja :} Employés pour la configuration de Kubernetes et dans le playbook Ansible.
    \item \textbf{Markdown :} Utilisé pour la création des différents fichiers Readme.
    \item \textbf{LaTex :} Utilisé pour la rédaction de ce rapport.
\end{itemize}

\subsection{Frameworks et bibliothèques JavaScript}\label{subsec:frameworks}
\begin{itemize}
    \item \textbf{NestJs :} Facilité le développement du back-end de l'application.
    \item \textbf{Angular :} Employé pour construire l'interface utilisateur et mettre en œuvre les tests.
    \item \textbf{TypeOrm :} Simplifié la gestion de la base de données.
    \item \textbf{PassportJs :} Utilisé pour la gestion de l'authentification.
    \item \textbf{Material UI :} Utilisé pour créer les composants de l'interface sur le front-end.
    \item \textbf{Apollo Angular :} Permis la communication avec l'API GraphQL du back-end.
    \item \textbf{Apollo Server :} A facilité la création de l'API GraphQL et la gestion des requêtes sur le back-end.
\end{itemize}

\subsection{APIs tierces}\label{subsec:apis-tierces}
\begin{itemize}
    \item \textbf{Open Street Map :} Cette API sera utilisée pour calculer les distances pour les covoiturages.
    \item \textbf{Spotify :} Potentiellement utilisée dans le futur.
    \item \textbf{Dicebear :} Utilisée pour générer des avatars pour les utilisateurs.
    \item \textbf{RestCountries :} Utilisée pour remplir la base de donnees avec tout les pays disponible.
\end{itemize}

\subsection{Autres ressources}\label{subsec:autres-ressources}
\begin{itemize}
    \item \textbf{Stack Overflow et Stack Exchange :} Sources précieuses de connaissances pour résoudre des problèmes spécifiques.
    \item \textbf{Documentations officielles :} Les documentations officielles d'Angular, NestJs, TypeOrm, Material UI, Apollo Angular, et K3S ont été des guides essentiels.
    \item \textbf{Chaîne YouTube de TechnoTim :} Principale source d'inspiration pour le cluster Kubernetes.
    Disponible sur : \url{https://www.youtube.com/@technotim}
    \item \textbf{Dépôt Git de TechnoTim :} Ce dépôt a servi de base pour la création du cluster Kubernetes.
    Disponible sur : \url{https://github.com/techno-tim/k3s-ansible}
\end{itemize}

Ces ressources ont joué un rôle crucial dans la réalisation de ce projet, contribuant à chaque étape du processus de développement.s de développement.