Dans le contexte actuel, les étudiants et les personnes qui aiment s'amuser en général organisent souvent des soirées ou des fêtes entre amis.
Pourtant, planifier et organiser de tels événements peut s'avérer difficile, en particulier lorsqu'il s'agit de fixer une date convenant à la majorité des participants et de gérer les dépenses.
Les défis augmentent lorsque le nombre de participants s'accroît.\\

Afin de résoudre ce problème, j'ai décidé de créer une application web qui facilite l'organisation de petites fêtes estudiantines.
Ce sujet présente un intérêt technique, car il implique la création d'une plateforme en ligne accessible à un large public.
De plus, il répond à un besoin réel des utilisateurs qui cherchent à simplifier l'organisation de leurs événements et à réduire les efforts nécessaires pour coordonner les participants.\\

L'application apportera une plus-value en proposant des fonctionnalités spécifiques pour gérer les aspects clés de l'organisation d'une fête, telles que la répartition des coûts,
la gestion des courses et le choix d'une date.
En outre, elle offrira un positionnement unique par rapport aux solutions existantes en offrant toutes ces fonctionnalités au sein d'une meme application.\\

Enfin, le sujet a été validé par un cours sondage au pres d'une vingtaine d'étudiants et une recherche d'application similaire deja existante, qui ont montré un intérêt pour une telle solution.
Ce travail représentera une charge de travail suffisante, estimée à environ 350 heures.